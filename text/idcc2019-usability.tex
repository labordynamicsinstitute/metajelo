Academic publishing outsources much of the content-related work to authors and subject matter editors. In order to be useful, the proposed package needs tools around it. We sketch out two such tools, and also address the role archives and repositories themselves play. 
\subsection{Metadata ingest}

We envision that the package be provided as a single file during the manuscript submission process by the author. This ensures that existing editorial workflow packages can seamlessly track the package, without needing upgrades to understand the content. Systems that do know how to ingest the information should do so, but are able to collect the information more efficiently. The package can be inspected by curation specialists and data editors and made available to reviewers as needed, and will follow the main document throughout the review process. 
           
\subsection{Creation by authors}
In order to create the package, we envision a simple website, which helps authors fill in the required information.\footnote{A demonstration project is available at \href{https://labordynamicsinstitute.github.io/metajelo-ui/}{labordynamicsinstitute.github.io/metajelo-ui}, archived as \citet{brandon_elam_barker_2021_4509001}.} Appropriate \ac{HCI} testing would need to be done to determine the optimal structure. However, the starting point is the DOI of the object being described, if available, or a bibliographic record, otherwise. From the DOI, a backend query to DataCite or CrossRef can reveal the hosting institution's institutionID. In turn, lookup in re3data or fairsharing.org will reveal elements of the institutional policies with regards to general access or preservation. Institutions often have multiple access policies and licenses, and which one applies to the object identified by the DOI may be hard to determine automatically. The author will be able to choose the appropriate license she consented to from a set of choices appropriate for the object and its hosting institution. In theory, all such information is provided through re3data, but failing to look up complete or accurate information, the author can also fill in the information manually. 
           
\subsection{Hosting by journals}
Journals are expected to post the package on their website, on the same landing page as the article itself. By doing so, the package itself can be parsed by appropriate tools.\footnote{A demonstration project is available at \href{https://labordynamicsinstitute.github.io/metajelo-web/}{labordynamicsinstitute.github.io/metajelo-web/}, archived as \citet{brandon_elam_barker_2021_4507862}.}  Naturally, more complex journal websites can include the contents in the page source code or in their \ac{CMS}. 

\subsection{Decentralizing the linkage architecture}
A final point is worth highlighting. When journals adopt the \metajelo package,  much information will be made available at a key point in the scientific cycle, when incentives are aligned: at the point of publication. By having authors themselves, possibly with help from the editors, create the linkage information (linking data and code archives to articles), having them describe what they know of access and retention policies at archives, creates information on thousands of articles every year, across hundreds of journals. This information can be harvested. Clearly, not all of the information will be accurate or consistent - but neither is the information currently being curated in centralized repositories of such information. Disambiguation algorithms will need to be deployed, and aggregation needs to allow for multiple (non-authoritative) answers. Facilities like Re3data will become aggregators instead of creators of such metadata. Our hypothesis is that the error rate in metadata will decline, but not disappear. 